This paper presents a Systematic Mapping (SM)  \cite{Petersen:2008} 
about the design of service-oriented applications in the presence of
non-functional requirements. SM is a  method for analyzing a field of interest (e.g., service oriented applications and NFR). The analysis can focus on
periodicity of publications organized by categories called facets combined to
answer  specific research questions  \cite{Budgen:2008}  that a scientist wishes to answer with quantitative data generated through the SM steps. 

In systems engineering, a non-functional requirement (NFR), also called qualities of a system, refer to the behaviour of a system. These criteria are not necessarily related to the output of the system or its application logic, but to the conditions of its  execution, its performance, and other properties (e.g., security, fault tolerance). NFR are also referred as
``constraints'', ``quality attributes'', ``quality goals'', ``quality of service
requirements'' or ``non-behavioural requirements''. In the case of service-based
applications, non-functional requirements concern the conditions in which the application is executed and also  constraints imposed by the services.

Associating non-functional requirements to services based applications can help to
ensure that the resulting application is compliant to the user requirements and
also with the characteristics of the services it uses. 
%As services are
%independent components, ensuring non-functional properties is a challenge.
 
The variability of terms about NFR comes from vocabularies of different domains, like software engineering, distributed systems, service oriented programming, etc. Therefore, the systematic mapping presented in this paper aims to identify the evolution of the area between 1998-2014 and the
relationship between concepts used for defining NFR and associating them to service oriented applications.
%From the results it will be possible to identify areas and works that can be
%done to improve the use and application of quality web services in distributed
%software development. Besides presenting an overview of the papers and
%researches that has been developed, and which are the main focus of these works.    
 
The remainder of this paper is organized as follows. Section \ref{sec:background} gives
the background about NFR and service oriented applications. Section \ref{sec:mappingprocess}  describes the systematic mapping
process and our research protocol, including the search strategy and selection of papers. Section \ref{sec:outcomes} presents and interpretes the analytics results.
Section \ref{sec:conclusions} concludes the paper and discusses research perspectives. 
