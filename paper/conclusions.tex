The mapping results presented can be the starting point
to motivate new studies, support the investigation of specific problems not sufficiently explored yet. The quantitative analysis provides an idea of the trends in service-based software development with NFR, including methodologies, languages and tools. The distribution of the papers that deal with NFR shows that they are addressed in different domains but  the vocabulary changes a lot and that there is a need of consensus, despite the existence of specifications like ISO/IEC 9126. When NFR are addressed at the level of the services they are related to QoS measures like economy or economic cost, availability, authentication requirements for contacting a service. NFR as defined by ISO/IEC 9126 are vast and papers address one or two at a time, particularly those related to the software engineering domain. Middleware solutions provide frameworks that consider different types of NFR but this concerns only the implementation stage of the software development process. This implies that the compliance between the design and the implementation might not be ensured.

With respect to the systematic mapping, we think that it requires a qualitative perspective that can be added by explicitly adding filtering and clustering criteria related to the provenance of the papers, the impact factor of the conference/journal where they appear, the reputation of the authors (given for example by their H factor), the institution and country of the authors. Without discarding the quantitative analysis, adding these criteria could increase the quality and value of the analysis. Similarly, we feel that choosing key words in the second phase of the methodology can be empirical, using vocabularies of the knowledge domain, could help to have a more representative choice. We are currently working in providing tools that can help to add quality to the systematic mapping method.
