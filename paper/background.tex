
This section introduces the vocabulary and concepts related to non-functional
requirement (NFR) of service-based applications.

A non-functional requirement often called quality of a system specifies a criterion  that characterises   the conditions in which it  operates.
According to \cite{MylopoulosBook99} there is no  formal definition or a
complete list of non-functional requirements. However, \cite{MylopoulosCN92} classifies NFRs  into consumer-oriented and
technically-oriented attributes.
 
In the area of software engineering, the term non-functional requirement refers to concerns that are not directly related to the functionality of the software.

Other authors say that NFR are ``\textit{requirements which are not specifically concerned with
the functionality of a system. They place restrictions on the product being developed and the 
development process, and they specify external constraints that the product 
must meet}'' \cite{Chung2009}.

Expressing and enforcing non-functional requirements for service-based
applications is a well-known problem with several associated existing solutions that have modeled
thoroughly them for providing middleware services. 

%......................................
\subsection{Adding NFR to service compositions}
%......................................

In Service-Oriented Computing~\cite{Papazoglou2007}, pre-existing services are
combined to build an application business logic.
The selection of services is usually guided by the \textit{functional} requirements of the application being developed~\cite{2,decastro1,PapazoglouH06}\footnote{Functional properties of a computer system are characterized by the effect produced by the system when given a defined input.}.
An important challenge of service-oriented development is  to ensure the alignment between the functional requirements imposed by the business logic and the functions actually being developed.

Functional properties are not the only  aspect in the software development process.
Non-functional requirements, such as data privacy, exception handling, atomicity  and, data persistence, need to be addressed  to fit in the application.
Adding non-functional requirements and respecting services constraints while composing services is a complex task that implies programming  protocols for instance authentication protocols to call a service, and atomicity (exception handling and recovery) for ensuring a true synchronization of the results produced by the service methods calls.

Even if service-oriented computing benefits from reuse, this  is usually guided only by functional requirements. 
Ideally, non-functional requirements should be considered in every phase of the software development.
Yet,  they are partially or rarely methodologically derived from the specification, being usually added once the code has been implemented. 
In consequence, the development process does not fully preserve the compliance and reuse expectations provided by the service oriented computing methods.


The literature stresses the need for methodologies and techniques for service oriented analysis and design 
%since they are the cornerstone in the development of meaningful service based applications
~\cite{Papazoglou2007}. 
Existing approaches argue that the convergence of model-driven software development, service orientation,   and  business processes improvement are key for developing accurate  software~\cite{watson}. 
Model Driven Development (MDD)  for software systems is mainly characterized by the use of models as a product~\cite{Selic03}.
These models are successively refined from abstract specifications into actual computer programs.

%\subsection{Problem space and solution space}

%......................................
\subsection{Models, methodologies and environments}
%......................................

General prupose methodologies do not fully consider NFR from the early phases of the (service) software process. 
 Most methods integrate them only after the application has been implemented. This leads to service based applications that are partialy specified and, thereby, partialy compliant with the requirements of the application.


The modeling of non-functional requirements from the early phases of the development can help the developer to produce applications that  can deal with the application context.


%......................................
\subsection{Related work}
%......................................
%{\em \bf Here other analysis of terms like the ones in PLacidos dissertation.}

\cite{Souza} proposes a classification of NFR as a result of a study concerning software methodologies for the construction of service-oriented systems. The classification  is organized in three layers representing: application modeling, services composition and services. The service composition layer serves as an integration layer between the service layer that exports methods and has associated constraints and characteristics; and the application layer that expresses requirements. 

At the application layer NFR can refer to business rules (e.g., only the user can publish data on her/his wall) and values for example, the email address is a string containing an “@” and a “.”. A value NFR expresses constraints about the way data and functions can be accessed and executed. For example accessing methods under security protocols.

Business NFR at the service layer concerns properties that are associated to services and defines how to call their exported operations (business properties). For example, response time, storage capacity (e.g., Dropbox service provides 5Giga free storage). Value constraints concern more on the conditions in which services can be used. For example, accessing to a function within an authentication protocol.

Finally, at the service composition layer gives an abstract view of the kind of properties exported by services that can be combined for providing NFR for a composition.  For example, confidentiality, authentication, privacy and access control can provide security at the service composition layer. 
\cite{ScCM10} proposes concepts and requirements for characterizing and analyzing existing approaches addressing NFR for service coordinations. Therefore they propose:
\begin{itemize}
\item	A meta-model that introduces for characterizing NFR according to the entity to which they are associated: attribute, concern, action, and activity. 
\item	Six requirements for studying NFR definition approaches for service coordinations: NFR specifications, NFR actions specification, Web service subjects specification, non functional attributes execution order specification, composite Web service subjects specification, stateful non functional constraint specification.

\item	Seven requirements for studying NFR enforcement approaches: separation of concerns, transparent integration of functional and NFR, quantification, superimposition, integration of NFR with distributed Web service, programming language independence, Web service composition support.
\end{itemize}

NFR concepts are close to those defined by \cite{Souza} but they are not organized into layers.  They correspond essentially to the service and service composition layers. The en-forcement requirements describe the way NFR are weaved to service coordinations. Exist-ing works respect all or a subset of these requirements. The chosen requirements have an impact in the way NFR are enforced at execution time.

