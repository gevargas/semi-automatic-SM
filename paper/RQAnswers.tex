
\bigskip

In the following sections we provide answers to the reaserch questions presented in section~\ref{sec:ResearchQuestions}.

%...............................................................................................................................................................
\subsection{RQ1: Which stages of  the service-based software development process have addressed NFR?}
%...............................................................................................................................................................

Analyzing the data presented in Figures~\ref{fig:Facets-Contribution-ProcessParadigm} and~\ref{fig:Facets-Paradigm-ContributionProcess}, we observe that NFR are considered at all stages of the software process for web applications.
However, most of the research efforts have focus on the design phase.
It is worth to notice that popular contributions are related to the organization of the software process, as well as middleware solutions.

Post-implementation phases (tests, validation and maintenance) are still to be better explored. 
Contrarily to our expectations, the testing activity is addressed by a very small number of papers.
This may indicate that either \textit{(i)} testing NFR compliance do not require specific techniques or \textit{(ii)} there is a research opportunity/challenge in this area.

Semantic tools (such as ontologies) are the preferred method for supporting NFR in web applications.
Although the usage of model-driven techniques is significant, we have not identified papers reporting on testing and validation using this approach. 
We believe that this is an area that deserves further investigation.

Our study shows a lack of testing NFRs using traditional software engineering methods for web applications.
This is not surprising since traditional testing techniques deal with functional requirements.

Another conspicuous absence is related to the support given by service composition languages to the testing and validation phases of software development. 
Some recent initiatives (such as~\cite{piSOD-M}) deals with this problem.



%...............................................................................................................................................................
\subsection{RQ2: What type of solutions have been proposed over the years to deal with NFR for service-based software?}
%...............................................................................................................................................................
Represent, specify and implement ...

%...............................................................................................................................................................
\subsection{RQ3 : Which is the scope of existing solutions for addressing NFR?}
%...............................................................................................................................................................
 Global/local and NFR types (one or several NFRs).
 
%\subsection{Research gaps and open issues}



